\section{Introduction}
The \textbf{monitor} program is a Qt-based program for monitoring
and controlling remote systems (typically robots) using key/value pairs
sent as text over UDP channels. The program takes a configuration
file describing the variables to be monitored and UI widgets viewing
those variables (or expressions based on them). Certain widgets 
can also send data to the remote.

\subsection{Requirements}
\begin{itemize}
\item Qt version 4
\item The Marble widget for displaying maps
\item Compiled under Ubuntu -- should work, but untested on anything else
\end{itemize}

\subsection{Compilation}
\begin{itemize}
\item Install Qt4
\item Install Marble (see \url{https://marble.kde.org/install.php}) --
under Ubuntu this entails installing the \texttt{marble} and \texttt{libmarble-dev}
packages
\item From the main directory, run \texttt{qmake-qt4} (or just \texttt{qmake} if
only Qt4 is installed) to build the makefile, then run \texttt{make}
\item Once built, test by running \texttt{./monitor -f exampleconfig}
\end{itemize}

\subsection{Receiving data}
Typically, the remote system sends lines of data which looks like this:
\begin{v}
time=13.0 x=1 y=2 someval=100
time=13.2 x=0
time=13.5 x=1 y=2
\end{v}
That is, space-separated key/value pairs containing a mandatory
timestamp. Values are all numerical (floating point).

\subsection{Sending data}
Some monitor widgets can also send data to the remote. This data
is also in space-separated key-value pairs.

\subsection{Running}
To run the monitor, use the command
\begin{v}
./monitor -f configfile
\end{v}
where \texttt{configfile} is the name of a valid configuration file.
